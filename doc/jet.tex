\documentclass[xelatex,hyperref={pdfpagelabels=false}]{beamer}

\usetheme{hbm}

\usepackage[ngerman]{babel}
\usepackage[babel,ngerman=guillemets]{csquotes}
\usepackage{fixltx2e}
\usepackage{fontspec}
\usepackage{xltxtra}
\usepackage{xunicode}
\usepackage{graphicx}

\usepackage{listings}
\usepackage{url}

\usepackage{tikz}
\usetikzlibrary{arrows,backgrounds,fit,decorations.pathmorphing,positioning,chains,calc}


\defaultfontfeatures{Mapping=tex-text}
\setmainfont{Myriad Pro}
\setsansfont{Myriad Pro}
\setmonofont{Courier Std}

\beamertemplatenavigationsymbolsempty

\title[jet] % short title, used for footer
{Funktionsweise von jet & jetcached}

\author{Gerhard Lipp}

\institute[T-GS/HBM]
{
    T-GS, Hottinger Baldwin Messtechnik GmbH\\
    Tel: 49 (0) 6151 8038344\\
    gerhard.lipp@hbm.com
}

\date{29. März 2012}

\subject{Informatik}

\begin{document}
{
\setbeamertemplate{footline}[hbm title]
\setbeamertemplate{title page}[hbm]
\begin{frame}
  \titlepage
\end{frame}
}

\setbeamertemplate{footline}[hbm]

\begin{frame}[fragile]{Was ist jet?}
\begin{block}{jet}
\begin{itemize}
\item Hierarchie
\item Knoten
\item Zustand
\item Methoden
\end{itemize}
Kurz: Semantik für zbus.
\end{block}
\end{frame}

\begin{frame}[fragile]{Was ist zbus?}
\begin{block}{jet url}
Bezeichnet eindeutig eine \emph{Methode} oder einen \emph{Zustand}.
Impliziert eine \emph{Hierachie}.
\end{block}
\begin{listings}{Beispiel}
system.uuid
\end{listings}
\end{frame}

\begin{frame}[fragile]{Was ist zbus?}
\begin{block}{zbus broker}
Der zbus broker ist ein Prozess, der die Nachrichten, die beim Aufrufen
von Methoden und absetzen von Notifications auftreten, verteilt bzw. routet.
Dazu verwaltet er:
\begin{itemize}
  \item Replier (Teilnehmer, die Methoden anbieten)
  \item Listener (Teilnehmer, die Notifications empfangen wollen)
\end{itemize}
Außerdem verwaltet er einen Pool von lokalen Socket URLs, die
Teilnehmern zur Verfügung gestellt werden können.
\end{block}
\end{frame}

\begin{frame}[fragile]{Welches Protokoll benutzt zbus?}
zbus basiert sein Protokol ausschließlich auf ZeroMQ multi-part
messages und simplen (unformatierten) Strings (für die Registry).
\end{frame}

\begin{frame}[fragile]{Motivation}
\begin{itemize}
  \item Funktionalität (in-)transparent über Prozesse verteilen
  \item Notifications über Modulgrenzen hinweg ermöglichen
  \item Effizienter sein als DBus
  \item Flexibler sein als DBus
\end{itemize}
\end{frame}

\begin{frame}[fragile]{Anforderungen an zbus Teilnehmer}
\begin{itemize}
\item zbus Teilnehmer müssen ZeroMQ ``sprechen''
\item zbus Teilnehmer müssen in der Lage sein, eine
  ZeroMQ Socketverbindung zum zbus Broker herzustellen. In der Regel
  wird eine TCP Verbindung benutzt
\end{itemize}
\end{frame}

\begin{frame}[fragile]{Methoden aufrufen}
%\begin{center}
\includegraphics[width=0.9\textheight,angle=270]{call_method.pdf}
%\end{center}
\end{frame}

\begin{frame}[fragile]{Notifications}
%\begin{center}
\includegraphics[width=\textheight,angle=270]{notifications.pdf}
%\end{center}
\end{frame}

\begin{frame}[fragile]{Features}
  \begin{itemize}
    \item Anmelden von Methoden, und ``subscriben'' von Notifications
      per Ausdruck (z.B.: ``example.*'')
    \item Erkennen von nicht eindeutigen Methodenaufrufen
    \item Batch-Notifications (in einer physicalischen Message)
    \item ``Externes'' Verbinden an broker (von anderen Netzwerkknoten
      / Devices)
  \end{itemize}
\end{frame}


\setbeamertemplate{title page}[hbm last page][Danke schön!]
\begin{frame}
  \titlepage
\end{frame}

\end{document}


